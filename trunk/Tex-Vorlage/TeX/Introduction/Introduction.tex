\chapter{Introduction}\label{secIntroduction}

Object Linking and Embedding for Process Control Unified Architecture, known as OPC UA is the most recent released industry standard from OPC Foundation, which compared with his predecessors is equipped with a list of charming new features, with whose help OPC UA is capable of developing a common communication interface for devices which participate in automation system. Meanwhile, the technology of smart card is widely used in information security fields of finance, communication, personal and government identification, payment. Therefore it is meaningful and promising to develop OPC UA standard compliant application on embedded smart card secure device, for the purpose of secure remote control, enterprise resource planning and etc. Since the storage and compute capacity of chip card is limited, OPC UA product will consist of two essential parts, namely client/server application code, realized as Android App or other application, and communication stack, realized as Javacard Applet based on Remote Application Management from GlobalPlatform. The implemented demonstration scenarios and corresponding analysis show the possibility of developing OPC UA standard compliant application on devices embedded with smart card to benefit customers. 

\section{Motivation}\label{secMotivation}

According to the \emph{Mobile Economy 2013} from \emph{Global System for Mobile Communications Association}, at the end of year 2013 there are over 3.2 billion mobile   subscribers in total, which means one half the population of the earth now enjoy the social and economic convenience brought by mobile technology. Moreover by year 2017 700 million new subscribers are expected to be added. And the number of mobile subscriber will reach 4 billion in 2018. Mobil technology opens nowadays a promising market. 

Mobile products  play an irreplaceable role at the heart of our  daily life. With the help of mobile technology, the user's world in many domains such as, education, financial transactions, health and etc. are inter-connected. Mobile users are enjoying the advantages of mobility. Services, like 24/7 monitored home security, full control over the management of home humidity and temperature, exist not only in science fiction film but also could be realized by today's technology.

At the same time, mobility in industry and business world is also a critical  assert, which can not only increase efficiency and productivity but also drive new revenue generation and competitive advantage. The most convicting example here is Machine to Machine communication, that is also referred as M2M technology. In M2M communication, machines which are usually embedded with smart cards exchange gathered data with each other to accomplish common task using wireless or wired networks. M2M technology is widely employed in different industry spheres such as factory automation, remote access control and sensor monitoring. It boosts the efficiency of corresponding processes, offers centralized service support and data management, minimizes system response time.

But in order to enjoy the aforementioned features, two tough issues must be resolved. First, how to achieve a common interface for the devices that build the system.  And second how to guarantee system security under different communication environments with various data complexity and customer's requirement.

\section{Solution Idea}\label{secSolutionIdea}

In this master thesis, I am going to address solutions for questions mentioned in  section \emph{Introduction and Motivation} and  design a smart home system for the purpose of demonstration. In this smart home system, home owner using smart phone is capable of experiencing 24/7 home security service, remotely managing inner home environment parameters and assigning access permissions. This system consists of smart phones with Universal Integrated Circuit Cards (UICC  smart card), digital door locks, electronic devices (such as coffee maker) and environment sensors. Moreover each device is equipped with smart card, which acts not only as secure token, that saves  user credentials, but also is in charge of communication management with other devices.

In particular, I will introduce the newly released industry automation standards object linking and embedding for processes control unified architecture(OPC UA standards) to build a common communication interface for devices that are mentioned above and design a OPC UA specification compliant communication stack on UICC smart card., whose duties are: creation and management communication between OPC client/server application, entity authentication and secure message exchange.

\section{Thesis Structure}
At first, in the second chapter I will present the fundamental technologies which will be frequently mentioned in this paper. Secondly the state of art, for instances mobile security,  home remote control technologies, Remote Application Management from GlobalPlatform will be introduced, which act together as cornerstone for my implementation scenario and give me inspiration materials for my thesis. In the fourth section,namely design phase, I am going to focus on UICC mobile security and base on UICC framework build a OPC UA standards scarifying communication stack as Javacard Applet with the help of GlobalPlatform  specifications, moreover the design of Android application which is introduced as OPC UA client application and a Smart Home web server that acts as Smart Home are going to be presented. In the next implementation chapter, I will present how my demonstration scenario can be realized. As sixth and seventh chapter,  test and performance analysis will be describe to show the reliability and security of my proposal.