\chapter{Introduction}\label{secIntroduction}

Object Linking and Embedding for Process Control Unified Architecture, known as OPC UA is the recently released industry standards set from OPC Foundation, which compared with its predecessors is equipped with a list of charming new features, with whose help OPC UA is capable of developing a common communication interface for devices manufactured by different vendors. Meanwhile, the technology of smart card is widely used in information security fields of finance, communication, personal as well as government identification and payment. Therefore it is meaningful and promising to develop OPC UA standard compliant applications on embedded smart cards secure devices, for the purpose of secure remote application    control, enterprise resource planning and etc. The applied OPC UA client and server application code builds a common connectivity and communication platform for other device specific applications. Meanwhile the employed smart card not only acts as security token storing credential information but also is in charge of peer identification as well as secure messaging by applying Remote Application/File Management over Http protocols standardized by GlobalPlatform. The in this thesis implemented demonstration scenario and corresponding analysis result prove the feasibility and reliability of my proposal. 

\section{Motivation}\label{secMotivation}

According to the report \emph{Mobile Economy 2013} from \emph{Global System for Mobile Communications Association}, at the end of year 2013 there are over 3.2 billion mobile subscribers in total, which means one half the population of the earth now enjoy the social and economic convenience brought by the mobile technology. Moreover by the year 2017, 700 million new subscribers are expected to be added. The number of mobile subscriber will reach 4 billion in 2018 \cite{last}. Mobil technology opens nowadays a promising market.

Mobile products play an irreplaceable and immense role at the heart of our daily life. With the help of mobile technology, the user's world in many domains such as, education, financial transactions, health and etc. are inter-connected. Mobile users are enjoying the advantages of mobility. Services, such as 24/7 monitored home security and customized control as well as management over housing devices, exist not only in science fiction film but also could be realized using present-day technologies.

At the same time, mobility is also a critical assert in industry and business world, which can not only increase efficiency and productivity but also drive new revenue generation and competitive advantage. The most convincing example here is the well known Machine to Machine communication technology, that is also referred as M2M technology. In a M2M communication system, machines are capable of  exchanging gathered data with each other using wireless or wired network connection, together accomplishing complex tasks and making intelligent scheduling decisions. M2M technology is widely employed in different industry spheres such as factory automation, remote access control and device monitoring. It boosts the efficiency of manufacturing processes, offers centralized service support and data management, minimizes system response time.

\section{Challenges}
But in order to enjoy the aforementioned advantages, two tough issues must be resolved. First one is, how to achieve a common communication and connectivity interface for devices manufactured by various vendors. And secondly since nowadays' systems are frequently exposed in a ubiquitous network and must confront a pervasive computing environment \cite{embedded_secure}, system designer must also concern about how to protect the system security in different communication environments with various data complexity and customer's requirement.

\section{Solution Idea}\label{secSolutionIdea}
In this thesis, I am going to present the solution solving the above mentioned challenges and design a Smart Home system for the purpose of proposal demonstration. In this Smart Home system, home owner with the help of a smart phone is capable of experiencing services such as, 24/7 home security, inner home environment management, assigning access permissions and so on. To be more specifically, the system consists of smart phones and various housing devices, such as digital door locks, coffee maker and environment sensors. Moreover each aforementioned device is equipped with an Universal Integrated Circuit Cards (UICC smart card), which acts not only as the secure token, that keeps user's credential information, but also is in charge of peer authentication, creating secure connection and managing communication with other devices by applying remote application and file management protocol from GlobalPlatform association.

In particular, I will introduce the newly released industry automation standards Object Linking and Embedding for Processes Control Unified Architecture (OPC UA standards) to build a common communication interface for devices that are mentioned above and design an OPC UA specifications compliant Java Card applet named \emph{CommunicationStack} on the UICC smart card. The \emph{CommunicationStack} provides services such as secure communication channel creation, entity authentication and secure messaging.

\section{Thesis Structure}
In the second chapter I will present the fundamental technologies which will be frequently mentioned in this paper. The third chapter is going to focus on the state of art technologies in the industry world, which could offer me inspirations and show me the best practices. In the fourth chapter, namely \emph{Implementation Scenario}, I will present the concept of my demonstration system and explain software structures as well as communication flows which are applied by my system. The following \emph{System Design} Chapter will concentrate on the design of system components. To be more precisely, I will discuss how I designed the Java Card applet \emph{CommunicationStack} and the Android application \emph{Smart Home App}. In the next \emph{Implementation Test} chapter, I will present some testing results of my demonstration system. As seventh chapter, \emph{System Security Analysis} will describe potential threads to my system as well as corresponding countermeasures, in order to prove the reliability and security of my proposal.