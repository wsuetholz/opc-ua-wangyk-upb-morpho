\chapter{Introduction}\label{secIntroduction}

Object Linking and Embedding for Process Control Unified Architecture, known as OPC UA is the most recent released industry standards set from OPC Foundation, which compared with his predecessors is equipped with a list of charming new features, with whose help OPC UA is capable of developing a common communication interface for devices manufactured by different vendors. Meanwhile, the technology of smart card is widely used in information security fields of finance, communication, personal and government identification, payment. Therefore it is meaningful and promising to develop OPC UA standard compliant application on embedded smart card secure device, for the purpose of secure remote control, enterprise resource planning and etc. Since the storage and compute capacity of a chip card is limited, I propose that the aforementioned application system should consist of two essential parts. Namely, the client/server application code, which is realized as for instance Android Application. And the communication stack, which is designed as a Javacard Applet applying Remote Application Management protocol from GlobalPlatform. The implemented demonstration scenarios and corresponding analysis in this thesis proof the feasibility of my proposal. 

\section{Motivation}\label{secMotivation}

According to the \emph{Mobile Economy 2013} from \emph{Global System for Mobile Communications Association}, at the end of year 2013 there are over 3.2 billion mobile   subscribers in total, which means one half the population of the earth now enjoy the social and economic convenience brought by mobile technology. Moreover by year 2017, 700 million new subscribers are expected to be added. The number of mobile subscriber will reach 4 billion in 2018. Mobil technology opens nowadays a promising market.

Mobile products play an irreplaceable and immense role at the heart of our daily life. With the help of mobile technology, the user's world in many domains such as, education, financial transactions, health and etc. are inter-connected. Mobile users are enjoying the advantages of mobility. Services, such as 24/7 monitored home security and customized control as well as management of housing devices, exist not only in science fiction film but also could be realized using present-day's technology.

At the same time, mobility is also a critical assert in industry and business world , which can not only increase efficiency and productivity but also drive new revenue generation and competitive advantage. The most convicting example here is the well known Machine to Machine communication technology, that is also referred as M2M technology. In M2M communication, machines  with the help of embedded chip cards are capable of  exchanging gathered data with each other using wireless or wired networks, together accomplishing complex tasks, making wise scheduler decisions and etc. M2M technology is widely employed in different industry spheres such as factory automation, remote access control and device monitoring. It boosts the efficiency of corresponding processes, offers centralized service support and data management, minimizes system response time.

But in order to enjoy the aforementioned features, two tough issues must be resolved. First, how to achieve a common interface for the devices that build the system.  And second how to guarantee system security under different communication environments with various data complexity and customer's requirements.

\section{Solution Idea}\label{secSolutionIdea}
In this  thesis, I am going to address solutions for the above mentioned questions and design a Smart Home system for the purpose of demonstration. In this Smart Home system, home owner with the help of a smart phone is capable of experiencing services such as, 24/7 home security, inner home environment management ,assigning access permissions and so on. To be more specifically, the system consists of smart phones and housing devices, e.g. digital door locks, coffee maker and environment sensors. Moreover each aforementioned device is equipped with an Universal Integrated Circuit Cards (UICC smart card), which acts not only as the secure token, that keeps user's credential information, but also is in charge of communication management with other devices.

In particular, I will introduce the newly released industry automation standards Object Linking and Embedding for Processes Control Unified Architecture(OPC UA standards) to build a common communication interface for devices that are mentioned above and design a OPC UA specification compliant communication stack Applet on the UICC smart card, whose duties are: creation and management communication between OPC client/server application, entity authentication and secure message exchange.

\section{Thesis Structure}
In the second chapter I will present the fundamental technologies which will be frequently mentioned in this paper. Second chapter is going to focus on the state of art technologies in the industry world, which could offer me inspirations and show me the best practices. In the fourth chapter, namely \emph{Implementation Scenario}, I will present the concept of my demonstration system and explain software structures as well as communication flows which are applied by my system. The Following \emph{System Design} Chapter will concentrate on the design of system components. To be more precisely, I will discuss how I design the Javacard Applet and the Android Application. In the next \emph{Implementation Test} chapter, I will present some testing results of my demonstration system. As seventh chapter, \emph{System Security Analysis} will describe potential threads to my system as well as corresponding countermeasures, in order to proof the reliability and security of my proposal.