% -*- TeX -*- -*- ENG -*- -*- UNIX -*-

\input{myDocumentClass} % Dokumentklasse bestimmen

\input{Constants} % Konstanten laden, z.B.: Begriffe wie Bachelor-Arbeit, Bachelor of Science, etc.

\usepackage[american]{babel}
%\usepackage[ansinew]{inputenc} %F�r deutsche Kodierung von Buchstaben �,�,�,� unter Windows (f�r Unix: 'latin1' statt 'ansinew') ansonsten muss man Folgendes schreiben:
% \"a f�r �, \"o f�r �, \"u f�r �, \ss f�r �.
%\usepackage{german} % unter anderem f�r bessere Silbentrennung

%#### My Input Data #######################################

% the following values have to be adapted for each thesis


% The following instruction defines the type of thesis or report, the following arguments are allowed:
% \ausarbeitungsTypBachelor		Bachelor's thesis
% \ausarbeitungsTypMaster			Master's thesis
% \ausarbeitungsTypDiplom			Diploma thesis
% \ausarbeitungsTypSeminar		Seminar report (Master's program)
% \ausarbeitungsTypProSeminar	Pro-Seminar report (Bachelor's program)
\def\ausarbeitungsTyp{\ausarbeitungsTypMaster}

\def\meinErstellungsdatum{July 2014} % day of completion, e.g. May 2009
\def\meinTitel{Leveraging Object Linking and Embedding for Process Control Unified Architecture
Standards with Smart Card Technology for Secure Applications and Services} % your title
\def\meinName{Yuankui Wang} % your name
\def\meineStrasseHausNr{Dessauer Str. 4} % street and house number
\def\meinePLZundOrt{33106 Paderborn} % zip code and city
\def\meinErstgutachter{Dr.rer.nat Simon Oberth\"ur} % first supervisor, applies to seminars, too
\def\meinZweitgutachter{Dr. Stefan Sauer} % second supervisor, irrelevant for seminars
\def\meinePDFStichwoerter{Technical articles, master thesis,Smart card, Secure application, Smart Home} % key words

\def\titelDesSeminars{Software Engineering for Software-Intensive Systems} % seminar topic, apllies to seminars only

%############################################################### % Parameter wie Autor, Titel, Datum, Gutachter, etc.

\usepackage{ifthen} % Paket f�r if-then-else-Konstrukte

\usepackage{verbatim}  % z.B. f�r comment-Umgebung
%\usepackage{natbib}    % f�r BibTeX mit dinat style (DIN 1505, Teil 2 und 3)
%\usepackage{bibgerm}   % f�r BibTeX mit deutschem style z.B. geralpha
%\usepackage{makeidx}   % f�r Index-/Stichwortverzeichnis
%\makeindex             % l�sst LaTeX die Indexeintr�ge sammeln
%\usepackage{epsfig}   % zum Einf�gen von EPS-Grafiken
\usepackage{array}     % weitere Hilfsmittel f�r Tabellen
%\usepackage{float}    % f�r weitere Gleitumgebungen au�er table und figure

\usepackage{setspace}  % f�r 1-fachen, 1,5-fachen oder 2-fachen Zeilenabstand

%\floatstyle{plain}
%\floatname{sourcecode}{Code-Fragment}
%\newfloat{sourcecode}{htbp}{loc}[chapter] % neue Gleitumgebung f�r Quellcode


\input{MathDefinitions} % einige mathematische Hilfskonstrukte und Vereinfachungen

\input{PageLayout} % Layout der Seiten, z.B. Kopf- und Fu�zeilendefinition


% ueberpruefen, ob wir pdflatex ausfuehren (geht nur bei Koma-Klassen)
\ifpdfoutput
{
% PDF wird genutzt
	
	\usepackage[pdftex]{graphicx,color}
	% eigene Farben f�r Links:
  %\definecolor{myLinkColor}{rgb}{0,0,.5}
  %\definecolor{myCiteColor}{rgb}{0,.5,0}
  %\definecolor{myFileColor}{rgb}{.5,0,0}
  %\definecolor{myURLColor}{rgb}{0,0,1}
  % Setzen aller Link-Farben auf Schwarz
  \definecolor{myLinkColor}{rgb}{0,0,0}
  \definecolor{myCiteColor}{rgb}{0,0,0}
  \definecolor{myFileColor}{rgb}{0,0,0}
  \definecolor{myURLColor}{rgb}{0,0,0}
  \usepackage[pdftex,%
  						pdftitle={\meinTitel},%
  						pdfauthor={\meinName},%
  						pdfkeywords={\meinePDFStichwoerter},% Stichw�rter
  						plainpages=false,%
  						pdfpagelabels,% Seitenzahl als z.B. 'ii (4 of 40)' anstatt '4 of 40' darstellen
  						colorlinks=true,%
  						linkcolor=myLinkColor,%
  						citecolor=myCiteColor,%
  						filecolor=myFileColor,%
  						urlcolor=myURLColor,%
  						bookmarks,% Lesezeichen erstellen
  						bookmarksnumbered, % Lesezeichen nummeriert wie im Inhaltsverzeichnis
  						breaklinks, % Zeilenumbr�che bei Links erlauben
  						%pdfpagelayout={TwoColumnRight}% zweiseitiges fortlaufendes Layout
  						]{hyperref}
  \pdfcompresslevel=9 %Kompressionslevel fuer Text und Grafiken
  \DeclareGraphicsExtensions{.pdf, .png, .jpg, .tif, .mps} % Dateiendungen f�r Grafikdateien, geordnet nach Priorit�t f�r automatische Auswahl der richtigen Datei, falls Endung nicht angegeben
}
{
% Kein PDF
  
  \usepackage{graphicx}
  \usepackage{color}
  \usepackage[hypertex, bookmarks,% Lesezeichen erstellen
  						bookmarksnumbered, % Lesezeichen nummeriert wie im Inhaltsverzeichnis
  						breaklinks, % Zeilenumbr�che bei Links erlauben
  						]{hyperref}
  \DeclareGraphicsExtensions{.eps,.ps} % Dateiendungen f�r Grafikdateien
}

\input{DocumentStructure} % der eigentliche Inhalt und die Struktur der Arbeit