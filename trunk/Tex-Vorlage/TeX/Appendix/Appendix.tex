\chapter{My Appendix}\label{secAppendixA}

The Appendix can be used to provide additional information, e.g. tables, figures, etc.


\section{Some Exemplary Uses of \LaTeX}

In the following a few exemplary uses of \LaTeX~commands are introduced.

\subsection{Figures}\label{secFigures}

You can reference figures in your text, e.g. Figure~\ref{figUniKralle} on Page~\pageref{figUniKralle}.

\begin{figure}[htb]
	\centering
	\includegraphics[scale=0.5]{Images/TitlePage/uni-logo}
	\caption{A single scaled image}
	\label{figUniKralle}
\end{figure}

You can even combine more than one image in a single float environment and reference each of them independently, e.g. Figures \ref{figCaseA} and \ref{figCaseB}.

\begin{figure}[h]
	\centering
  \begin{minipage}[t]{.47\textwidth}
    \centering
    \includegraphics[width=4.3cm]{Images/TitlePage/uni-logo} 
    \caption{Case A}
    \label{figCaseA}
  \end{minipage}%
  \hfill
  \begin{minipage}[t]{.47\textwidth}
    \centering
    \includegraphics[width=6cm]{Images/TitlePage/uni-logo}
    \caption{Case B}
    \label{figCaseB}
  \end{minipage}
\end{figure}

\begin{figure}[t]
	\centering
	\includegraphics[width=\textwidth]{Images/TitlePage/uni-logo}
	\caption[Caption for the list of figures]{A single figure scaled to fill the page's space given for the text area.}
	\label{figUniKralleLarge}
\end{figure}


\subsection{Tables}

You can reference tables analogously to figures (see Section~\ref{secFigures}), e.g. Table~\ref{tabValues} on Page~\pageref{tabValues}.

\begin{table}[tb]
  \caption{A Table with some values}
  \label{tabValues}
  \centering
  \vspace{1ex}
		\begin{tabular}{l|l|l|l|l}
			Label & Amount of & \multicolumn{3}{c}{Values}\\
			& something & Case 1 & Case 2 & Case 3\\
			\hline
			\hline
			X & 100\% & 39.59\% & 86.47\% & 87.82\%\\
			\hline
			Y & 100\% & 38.03\% & 84.91\% & 86.26\%\\
			\hline
			Z & 71.88\% & 34.78\% & 97.83\% & 97.83\%\\
			\hline
		\end{tabular}
\end{table}


\subsection{References and Quotation}

You can cite books and other scientific documents, e.g. books about \LaTeX~ \cite{Latex4, Latex3, Latex1, Latex2} (the last three are in German).
Quotations have to be indented adequately, e.g.:

\begin{quotation}
	"\LaTeX~is great!"
\end{quotation}

\noindent\textbf{Attention!} When using citations in \LaTeX~consider that you have to compile your \LaTeX~file up to 4 times until all references are correctly set in your generated document.


\subsection{Tools and Environments}

In case of using Windows, the \LaTeX -environment MiKTeX \cite{MikTex} is a good choice. In this case the editor TeXnicCenter \cite{TexnicCenter} can be used to edit your text.